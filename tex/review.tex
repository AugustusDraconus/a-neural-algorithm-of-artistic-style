%% define the main template style for the document
\documentclass{article}

%\usepackage{nips_2017}
% to compile a camera-ready version, add the [final] option, e.g.:
\usepackage[final]{nips_2017}

\usepackage[utf8]{inputenc}
\usepackage[T1]{fontenc}
\usepackage[pagebackref=true]{hyperref}
\usepackage{url}
\usepackage{nicefrac}
\usepackage{microtype}
\usepackage[nolist,nohyperlinks]{acronym}

%% Import general packages
\usepackage{
  amsmath, amssymb, amsfonts,
  algorithmic, textcomp, listings,
  graphicx, subfig,
  booktabs, longtable
}

%% start the document, the Markdown parser takes over from this point
\begin{document}
\title{Review: A Neural Algorithm of Artistic Style}

\author{
    James C. Kauten \\
    Department of Software Engineering \\
    Auburn University \\
    Auburn, AL 36832 \\
    \texttt{jck0022@auburn.edu} \\
    \And
    Behnam Rasoolian \\
    Department of Industrial Engineering \\
    Auburn University \\
    Auburn, AL 36832 \\
    \texttt{bzr0014@auburn.edu} \\
}

\maketitle

\hypertarget{paper-summary}{%
\section{Paper Summary}\label{paper-summary}}

In their arXiv preprint, \textit{A Neural Algorithm of Artistic Style},
\cite{gatys2016image} prove a level of separability between the
\textit{content} of an image, and the \textit{style} that characterizes it.
Using a \ac{CNN} trained to classify images on the ImageNet benchmark, they
transfer the style of famous works of art, onto the content of arbitrary
photographs. To do so, they define loss functions between the activation maps
of various layers in the network that measure either content, or style loss.
Minimizing the joint loss between a white noise image $\textbf{x}$ and both a
content image $\textbf{p}$ and style image $\textbf{a}$ transfers the global
features of $\textbf{p}$, with the local styles of $\textbf{a}$ onto the white
noise image $\textbf{x}$. Put simply, their algorithm paints photographs using
arbitrary works of art as a palette.

\subsection{Representation}

\subsubsection{Content}

This work relies heavily on the understanding of convolutional layers. As a
collection of image filters, each layer extracts unique features from its
input image. As such, \cite{gatys2016image} postulate that as the layer depth
increases, the network cares more about the \textit{content} of the image.
That is to say, deeper layers have a more specific understanding of what
composes an image, whereas the shallower layers primarily understand the image
as raw pixels. This leads to their definition of
\textit{content representation} as the activations from deep layers in the
network.

To objectively measure the difference of content between two images,
\cite{gatys2016image} define a loss function $\mathcal{L}_{content}$. Given a
content image $\textbf{p}$, a white noise image $\textbf{x}$, and an arbitrary
layer $l$, the activations at $l$ for $\textbf{p}$ and $\textbf{x}$ are
defined as $P^l$ and $F^l$ respectively. The $\mathcal{L}_{content}$ loss then
is the squared euclidean distance between $P^l$ and $F^l$:

% TODO: note the M_l and N_l variables in the above paragraph
\begin{equation}
\mathcal{L}_{content}(\mathbf{p}, \mathbf{x}, l) =
\frac{1}{2} \sum_{i=1}^{N_l}\sum_{j=1}^{M_l}{(F^l_{ij} - P^l_{ij})^2}
\end{equation}

\subsubsection{Style}

Much like the content representation, style representation relies on the
feature responses of particular layers in the \ac{CNN}. However, this
representation uses a different feature space. Converting each activation
map to a \textit{gram matrix}, allows the extraction of just the
\textit{texture} from a given image. It does so by computing the correlations
between different filters in an arbitrary convolutional layer $l$. More
simply, the gram matrix $G^l$ for an activation map $F^l$ is the inner product
of feature maps:

\begin{equation}
G_{i j}^l = \sum_{k}^{M_l} F_{i k}^l F_{j k}^l
\end{equation}

With a new feature space representation of raw texture, \cite{gatys2016image}
define an additional loss function $\mathcal{L}_{style}$ between an artwork
image $\textbf{a}$, and a white noise image $\textbf{x}$ for some
convolutional layer $l$. First, the activations at $l$ for $\textbf{a}$ and
$\textbf{x}$ are transformed to their respective gram matrices $A^l$, and
$G^l$. Then, much like the content loss, we define $\mathcal{L}_{style}$ as
the squared euclidean distance between the gram matrices $A^l$, and $G^l$:

\begin{equation}
E_l =
\frac{1}{4 N_l^2 M_l^2}
\sum_{i=1}^{N_l}\sum_{j=1}^{M_l}
(G^l_{ij} - A^l_{ij})^2
\end{equation}

\begin{equation}
\mathcal{L}_{style}(\mathbf{a}, \mathbf{x}, l) = \sum_{l=0}^L w_l E_l
\end{equation}


\hypertarget{strengths}{%
\section{Strengths}\label{strengths}}

\hypertarget{weaknesses}{%
\section{Weaknesses}\label{weaknesses}}

\hypertarget{future-work}{%
\section{Future Work}\label{future-work}}

\hypertarget{qa}{%
\section{Questions \& Answers}\label{qa}}

\bibliographystyle{my-unsrtnat}
\bibliography{references}

% a collection of Acronyms
\begin{acronym}
\acro{CNN}{Convolutional Neural Network}
\end{acronym}

\end{document}
